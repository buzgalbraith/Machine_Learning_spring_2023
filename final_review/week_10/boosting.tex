\documentclass{article}
\usepackage[utf8]{inputenc}
\title{Lecture 10 boosting }
\author{wbg231 }
\date{December 2022}
\newcommand{\R}{$\mathbb{R}$}
\newcommand{\B}{$\beta$}
\newcommand{\A}{$\alpha$}
\newcommand{\D}{\Delta}

\newcommand{\avector}[2]{(#1_2,\ldots,#1_{#2})}
\newcommand{\makedef}[2]{$\textbf{#1}$:#2 }
\usepackage{tikz,graphicx,hyperref,amsmath,amsfonts,amscd,amssymb,bm,cite,epsfig,epsf,url}

\begin{document}

\maketitle

\section{motivation}
\begin{itemize}
\item recall in ada boosting we learn weak learners $g_m(x)$ and weights $\alpha_{m}$ and use those to get our final predictor $G(x)=sing(\sum_{i=1}^{m}\alpha_mG_m(x))$
\item why not just learn $G(x)$ directly? 
\item well it is a linear sum of weak learners that are not nesscarrily linear so we learn it sequentially in ada boost
\section*{nonlinear regression}
\item we can find wide types of data by fitting a linear combination of transformations to the input $$f(x)=\sum_{m=1}^{m}v_{m}h_m(x)$$ where $h_m$ is called a basis function such that $$h_1\cdots h_m:X\rightarrow \mathbb{R}$$
\item example polynomial regression $h_i(x)\in \{x^{i}:i\in \mathbb{Z}\}$ 
\item we can fit this with standard linear models if our basis functions are fixed ahead of time all we are learning are weighting factors  
\subsection*{adaptive bayes function model}
\item what if we want to learn the basis functions
\item we define a base hypotheses space $\mathcal{H}:X\rightarrow \mathbb{R}$ so all scalar values function
\item an \textcolor{blue}{adaptive basis function expansion} over $\mathcal{H}$ is an ensample model $$f(x)=\sum_{m}^{M}v_mh_m(x)$$ where $v_m\in \mathbb{R}, h_{m}\in \mathcal{H}$
\item so then we can combine these to get a new hypotheses space $$\mathcal{F}_{M}=\{\sum_{m}v_mh_m(x)|v_m\in \mathbb{R}, h_m\in \mathcal{H}\forall m\in[1...m]\}$$
\item so we our objective is $$j(v_1\cdots v_m, h_1\cdots h_m)=\frac{1}{n}\ell(y_i,f(x_i))=\frac{1}{n}\ell(y_i,\sum_{m=1}^{M}v_mh_m(x))$$
\item if we want to optimize this sometimes we can use gradient descent or find a closed form (but that is not always the case )
\item in cases where we can not differentiate we can try using a greedy algorithm similar to ada boost
\subsection*{gradient boosting }
\item applies when ever our loss function is sub differentiable wrt our training predictions $f(x_i)$ we can do regressions with the hypothesis base space $\mathcal{H}$
\section*{forward Stagewise adaptive modeling}
\item to recap our goal is to find the model $$f(x)=\sum_{m}v_mh_m(x)$$ that is a weighed sum of basis functions given some loss function
\item we do this by greedily fitting one function at a time without adjusting previous functions "forward Stagewise"
\item so after $m-1$ stages we will have $$f_{m-1}=\sum_{i=1}^{m-1}v_ih_i$$
\item and then at the mth round we are trying to find the basis function $h\in \mathcal{H}$ and $v_m>0$ such that $$f_m=f_{m-1}+v_mh_m=c+v_mh_m$$ improves our loss as much as possible 
\item so this is what our algorithm looks like \\ \includegraphics*[width=10cm]{images/Screenshot 2023-05-14 at 10.18.27 PM.png}
\item we are going to set our loss function as exponential ie $\ell(y,f(x))=e^{-yf(x)}$
\item and assume our $\mathcal{H}=\{h:x\rightarrow \{-1,1\}\}$ that is our base functions are binary classifiers
\item   \includegraphics*[width=10cm]{images/Screenshot 2023-05-14 at 10.26.42 PM.png}
\item so i mean the above is mainly algebra i am not sure if there is that much to it except for a helpful way to re-write the objective
\item there is a lot of kind of messy algebra in this section that i am not sure is super useful
\item the real take away is basically this a generalization of ada boost
\item in practice this has a robustness issue since ada boost is not robust to outliers
\subsection*{review}
\item so far we have seen that using a basis function to obtain nonlinear model $f(x)=\sum_{m=1}v_mh_m(x)$ if you know the basis functions
\item could use adaptive basis function models if you do not know the basis. 
\item and forward stage wise additive modeling greedily fits $h_m$ to minimize average los 
\item but fsam only works for some loss functions 
\item we need a more general model 
\section*{gradient boosting / any boost}
\subsection*{FSAM with squared loss}
\item our objective function is $$j(v,h)=\frac{1}{n}\sum_{i=1}^{n}(y_i-(f_{m-1}(x_i)+vh(x_i)))^2 $$ if $\mathcal{H}$ is closed under scaling then we can just set $v=1$ adn maximize and the model will adjust it's self 
\item doing v=1 yields $$j(h)=\frac{1}{n}\sum_{i}([y_i-f_{m-1}(x_i)-h(x)])^{2}$$ this si equivalent to fitting function residuals with least squares regression
\item  so we can think of this as sequentially building models such that each one minimizes least squared residuals with the ones before set 
\subsection*{interpret the residual}
\item so our objective $J(f)=\frac{1}{n}\sum_{i=1}(y_i-f(x_i))^2=\frac{1}{n}\sum_{i=1}(y_i-f_{m-1}(x)-f_m(x_i))^2$
\item so we can see that $\frac{\partial J(j)}{\partial f(x_i)}=-2(y_i-f(x_i))$ this gradient with respect to f is saying how should we try to change the output of f to minimize square loss
\item so in other words our residual is the gradient 
\item so what we are doing at each step is learning $h\in \mathcal{H}$ to fit the residual $$f\leftarrow f+vh$$
\subsection*{functional gradient descent}
\item we want to minimize our objective $$J(f)=\sum_{i=1}^{n}\ell(y_i,f(x_i))$$
\item not that $j(f)$ only depends at f evaluated at n training points that is $f=(f(x_1)\cdots f(x_n))^{t}$ so treating these as parameters we can write $$j(f)=\sum\ell(y_i,f_i)$$ 
\item the negative gradient $-g$ is the vector of partial darivtives of the y with respect to $f_i=f(x_i)$
\item with gradient descent the final predictor will be $$f_0+\sum v_t(-g_t)$$ that is at every step we are updating our parameters in the direction of the gradient 
\item the unconstrained step direction $-g$ is called the “pseudo residual"
\item so we only have h points which to use to estimate $h\in \mathcal{H}$ so we do projected least squares regression where $$min_{h\in \mathcal{H}}\sum(-g_i-h(x_i))^2$$ 
\subsection*{recap}
\item so we have the following objects 
\begin{enumerate}
    \item our objective function $$j(f)=\sum\ell(y_i,f(x_i))$$
    \item unconstrained gradient $g\in \mathbb{R}^{n}$ wrt $f=(f(x_1)\cdots f(x_n))^t$ (so this is a true gradient of the loss function with respect to what we have learned so far )
    \item then we have the projected negative gradient$h\in \mathcal{H}$ so that is $$h=argmin_{h\in \mathcal{H}}\sum_{i=1}^{n}(-g_i-h(x_i))^2$$
    \item and we update our function at each step acording to this projected gradient $f\leftarrow f + vh$
\end{enumerate}
\item \includegraphics*[width=10cm]{images/Screenshot 2023-05-14 at 11.17.34 PM.png}
\subsection*{binomial boost with logistic loss}
\item recall that logistic loss with $y\in [-1,1]$ 
 $$\ell(y,f(x))=log(1+e^{-yf(x)})$$
\item we can write it out and find our pseudo residual then plug that into our prediction function
\section*{gradient tree boosting}
\item one commong from of gradient boosting machine is to set the hyptoehsis space = to regression tree of a certian size
\item  boosting is good at resisting overfitting 
\item since there is implicit feature selection. we are greedtly selecting the best features (weak learners)
\item also overfit resilient because as training goes on the impact of change becomes localized 
\item still can overfit tho 
\item could also try doing stochastic gradient boosting where we do the same process just as stochastic gradient descent 
\item this can speed things up and also help prevent overfitting.
\item 
\end{itemize}
\end{document}